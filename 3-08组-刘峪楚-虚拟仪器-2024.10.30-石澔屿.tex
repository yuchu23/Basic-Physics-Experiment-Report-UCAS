% 本模板根据中国科学院大学本科生公共必修课程《基础物理实验》Word模板格式编写
% 本模板由Shing-Ho Lin和Jun-Xiong Ji于2022年9月共同完成, 旨在方便LaTeX原教旨主义者和被Word迫害者写实验报告, 避免Word文档因插入过多图与公式造成卡顿. 
% 如有任何问题, 请联系: linchenghao21@mails.ucas.ac.cn
% This is the LaTeX template for experiment report of Experimental Physics courses, based on its provided Word template. 
% This template is completed by the joint collabration of Shing-Ho Lin and Junxiong Ji in September 2022. 
% Adding numerous pictures and equations leads to unsatisfying experience in Word. Therefore LaTeX is better. 
% Feel free to contact us via: linchenghao21@mails.ucas.ac.cn

\documentclass[11pt]{article}

\usepackage[a4paper]{geometry}
\geometry{left=2.0cm,right=2.0cm,top=2.5cm,bottom=2.5cm}

\usepackage{ctex} % 支持中文的LaTeX宏包
\usepackage{amsmath,amsfonts,graphicx,subfigure,amssymb,bm,amsthm,mathrsfs,mathtools,breqn} % 数学公式和符号的宏包集合
\usepackage{algorithm,algorithmicx} % 算法和伪代码的宏包
\usepackage[noend]{algpseudocode} % 算法和伪代码的宏包
\usepackage{fancyhdr} % 自定义页眉页脚的宏包
\usepackage[framemethod=TikZ]{mdframed} % 创建带边框的框架的宏包
\usepackage{fontspec} % 字体设置的宏包
\usepackage{adjustbox} % 调整盒子大小的宏包
\usepackage{fontsize} % 设置字体大小的宏包
\usepackage{tikz,xcolor} % 绘制图形和使用颜色的宏包
\usepackage{multicol} % 多栏排版的宏包
\usepackage{multirow} % 表格中合并单元格的宏包
\usepackage{makecell} % 单元格中换行的宏包
\usepackage{diagbox} % 表格斜线的宏包
\usepackage{pdfpages} % 插入PDF文件的宏包
\RequirePackage{listings} % 在文档中插入源代码的宏包
\RequirePackage{xcolor} % 定义和使用颜色的宏包
\usepackage{wrapfig} % 文字绕排图片的宏包
\usepackage{bigstrut,multirow,rotating} % 支持在表格中使用特殊命令的宏包
\usepackage{booktabs} % 创建美观的表格的宏包
\usepackage{circuitikz} % 绘制电路图的宏包
\usepackage{amsmath}

\definecolor{dkgreen}{rgb}{0,0.6,0}
\definecolor{gray}{rgb}{0.5,0.5,0.5}
\definecolor{mauve}{rgb}{0.58,0,0.82}
\lstset{
  frame=tb,
  aboveskip=3mm,
  belowskip=3mm,
  showstringspaces=false,
  columns=flexible,
  framerule=1pt,
  rulecolor=\color{gray!35},
  backgroundcolor=\color{gray!5},
  basicstyle={\small\ttfamily},
  numbers=none,
  numberstyle=\tiny\color{gray},
  keywordstyle=\color{blue},
  commentstyle=\color{dkgreen},
  stringstyle=\color{mauve},
  breaklines=true,
  breakatwhitespace=true,
  tabsize=3,
}

% 轻松引用, 可以用\cref{}指令直接引用, 自动加前缀. 
% 例: 图片label为fig:1
% \cref{fig:1} => Figure.1
% \ref{fig:1}  => 1
\usepackage[capitalize]{cleveref}
% \crefname{section}{Sec.}{Secs.}
\Crefname{section}{Section}{Sections}
\Crefname{table}{Table}{Tables}
\crefname{table}{Table.}{Tabs.}

\setmainfont{Times New Roman}
\setCJKmainfont{黑体}
\setCJKsansfont{宋体}
\setCJKmonofont{仿宋}
\punctstyle{kaiming}
% 偏好的几个字体, 可以根据需要自行加入字体ttf文件并调用

\renewcommand{\emph}[1]{\begin{kaishu}#1\end{kaishu}}

\newcommand*{\unit}[1]{\mathop{}\!\mathrm{#1}}
\newcommand*{\dif}{\mathop{}\!\mathrm{d}}%微分算子 d
\newcommand*{\pdif}{\mathop{}\!\partial}%偏微分算子
\newcommand*{\cdif}{\mathop{}\!\nabla}%协变导数、nabla 算子
\newcommand*{\laplace}{\mathop{}\!\Delta}%laplace 算子
\newcommand*{\deriv}[2]{\frac{\mathrm{d} #1}{\mathrm{d} {#2}}}
\newcommand*{\derivh}[3]{\frac{\mathrm{d}^{#1} #2}{\mathrm{d} {#3^{#1}}}}
\newcommand*{\pderiv}[2]{\frac{\partial #1}{\partial {#2}}}
\newcommand*{\pderivh}[3]{\frac{\partial^{#1} #2}{\partial {#3^{#1}}}}
\newcommand*{\mcelsius}{\unit{\prescript{\circ}{}C}}
%改这里可以修改实验报告表头的信息
\newcommand{\experiName}{虚拟仪器在物理实验中的应用}
\newcommand{\supervisor}{石澔屿}
\newcommand{\name}{刘峪楚}
\newcommand{\studentNum}{2023K8009929030}
\newcommand{\class}{3}
\newcommand{\group}{08}
\newcommand{\seat}{2}
\newcommand{\dateYear}{2024}
\newcommand{\dateMonth}{10}
\newcommand{\dateDay}{30}
\newcommand{\room}{702}
\newcommand{\others}{$\square$}
%% 如果是调课、补课, 改为: $\square$\hspace{-1em}$\surd$
%% 否则, 请用: $\square$
%%%%%%%%%%%%%%%%%%%%%%%%%%%

\newcommand{\chapter}[2]{\begin{center}\bf\Large{第#1部分\quad #2}\end{center}}

\begin{document}

%若需在页眉部分加入内容, 可以在这里输入
% \pagestyle{fancy}
% \lhead{\kaishu 测试}
% \chead{}
% \rhead{}

\begin{center}
    \LARGE \bf 《\, 基\, 础\, 物\, 理\, 实\, 验\, 》\, 实\, 验\, 报\, 告
\end{center}

\begin{center}
    \noindent \emph{实验名称}\underline{\makebox[25em][c]{\experiName}}
    \emph{指导教师}\underline{\makebox[8em][c]{\supervisor}}\\
    \emph{姓名}\underline{\makebox[6em][c]{\name}} 
    % 如果名字比较长, 可以修改box的长度"6em"
    \emph{学号}\underline{\makebox[10em][c]{\studentNum}}
    \emph{分班分组及座号} \underline{\makebox[5em][c]{\class \ -\ \group \ -\ \seat }\emph{号}} (\emph{例}:\, 1\,-\,04\,-\,5\emph{号})\\
    \emph{实验日期} \underline{\makebox[3em][c]{\dateYear}}\emph{年}
    \underline{\makebox[2em][c]{\dateMonth}}\emph{月}
    \underline{\makebox[2em][c]{\dateDay}}\emph{日}
    \emph{实验地点}\underline{{\makebox[4em][c]\room}}
    \emph{调课/补课} \underline{\makebox[3em][c]{\others\ 是}}
    \emph{成绩评定} \underline{\hspace{5em}}
    {\noindent}
    \rule[8pt]{17cm}{0.2em}
\end{center}

\section{实验目的}

\begin{enumerate}

  \item 了解虚拟仪器的概念; 

  \item 了解图形化编程语言 LabVIEW, 学习简单的 LabVIEW 编程; 
  
  \item 完成伏安法测电阻的虚拟仪器设计. 

\end{enumerate}

\section{实验仪器与用具}

计算机(含Windows操作系统), LabVIEW 2014, NI ELVIS II+, 导线若干, 元件盒一个(包括$100 \Omega$ 标准电阻一个, 待测电阻$1 \unit{k\Omega}$ 和 $51 \unit{\Omega}$ 各一个, 稳压二极管一个), 热电偶等元件. 

\section{实验原理}

\begin{enumerate}

\item 虚拟仪器的硬件和软件

硬件:个人电脑(PC机),美国国家仪器公司的教学实验室虚拟仪器套件(NI ELVISII+)和自带的原型板。

软件:LabVIEW。LabVIEW将计算机数据分析和显示能力与仪器驱动程序整合在一起,为针对仪器的编程提供了很大的便利。LabVIEW采用了图形化编程语言,编程过程以设计流程图的形式进行,即使是初学者也能够迅速入门。

虚拟仪器程序:使用LabVIEW开发平台编制的虚拟仪器程序简称为VI。VI包括三个部分:前面板、程序框图和图标/连线板。前面板用于设置输入数值和显示输出量,相当于真实仪表的前面板。前面板上的图标分为两类:输入类(Controls,用于输入)和显示类(Indicators,用于输出),例如开关、旋钮、按钮、图形和图表等。程序框图相当于仪器的内部功能结构,其中的端口用来传递数据到前面板的输入对象和显示对象,节点用于实现函数和功能子程序的调用,图框用于实现结构化程序控制命令,而连线则代表了程序执行过程中的数据流。

\item 创建一个温度测量程序

首先是创建一个模拟温度测量程序. 假设有一个传感器, 其输出电压和温度成正比, 用它编写一个模拟温度测量的程序. 假设当温度为 $80 \unit{^\circ F} $ 时, 传感器输出电压为 $0.8\unit{V}$, 那么我们可以编写程序, 根据电压计算温度, 并且给出摄氏度和华氏度两种显示.
	
然后是使用真实热电偶元件创立一个温度测量程序. 假设传感器的输出电压和温度成正比. 例如, 当温度为华氏 80 °F 时, 传感器输出电压为 0.8V, 则可以利用程序根据电压计算温度.
	
\item 创建一个电压输出和采集的程序

本实验通过编写输出/输入通道, 两个停止按钮, 可以手动调整的输出电压与测量到的电压, 连接电路, 使用 While 循环使得程序每 $100\unit{ms}$ 输出/测量一次电压, 并且在电路板上连接好两根导线, 得到的结果就可以随时返回前面板.

\item 利用虚拟仪器测量伏安特性

本实验中利用一个模拟输出通道为整个测量电路供电, 利用两个模拟输入通道分别测量总电压和标准电阻上的电压; 利用测量得到的电压数值和标准电阻数值就可以得到电路中的电流以及待测电阻上的电压. 在程序控制下, 电路电压由 $0\unit V$ 开始逐渐增加到设定电压, 电压每改变一次, 测得一组电压电流值, 最后得到一个数组, 经过线性拟合后就可以得到待测电阻值. 测量电路图和原理图如图1所示:

\begin{figure}[H]
    \centering
    \subfigure[前面板]{\includegraphics[height=5cm]{实验四电路原理图.png}}\hspace{0.5cm}
    \subfigure[程序框图]{\includegraphics[height=5cm]{实验四电路图.jpg}}
    \caption{实验四: 利用虚拟仪器测量伏安特性}
\end{figure}

\end{enumerate}

\section{实验内容}

\begin{enumerate}

\item 初步熟悉 LabVIEW 开发环境的基本操作和编程方法
  
在 LabVIEW 2014 中,通过选择"文件"菜单中的"新建V1",可以打开前面板和程序框图。使用快捷键 Ctrl + T 可以并排打开两个窗口,方便编程。在前面板的"查看"菜单中打开"控件选板"和"工具选板",可以从"工具选板"中选择"自动选择"工具,方便操作。在"控件选板"中可以新建"温度计"并将其显示出来。
  
在程序框图窗口中,通过在"查看"菜单中打开"函数选板",可以显示函数选板。利用"函数选板"新建"加法",并尝试为之连线。
  
在开始下列实验之前,请确保打开面板上的两个开关,即 ELVIS 电源(在仪器后面)和原型板电源(在仪器上面的右上方)。

\item 创建一个模拟温度测量程序
  
在 LabVIEW 中,我们可以新建一个空白 VI,并在前面板中添加温度计、垂直滑动杆开关、数值显示控件和数值输入控件,并对它们进行重命名。我们还可以在适当的位置添加两个标签。
  
在程序框图中,我们可以新建减法、乘法、除法和选择模块,并将它们与之前创建的四个模块放在合适的位置,并用线路将它们连接起来。我们可以右键单击减法、乘法和除法模块的空余输入位置,选择创建"常量",并分别输入 $32$、$100$ 和 $1.8$ 三个数字。
  
接下来,我们可以点击连续运行按钮,使程序进入连续运行模式。在 "采集的电压" 输入框中输入一些数值(比如在 $0.5 \sim 2$ 之间的任意值),选择温度值单位,并观察程序的运行情况。最后,我们可以再次点击连续运行按钮,停止程序的运行,并保存并关闭程序。
  
前面板和程序框图如下图所示:
  
\begin{figure}[H]
    \centering
    \subfigure[前面板]{\includegraphics[height=5cm]{实验二前面板.jpg}}\hspace{0.5cm}
    \subfigure[程序框图]{\includegraphics[height=5cm]{实验二程序框图.jpg}}
    \caption{实验二: 创建一个模拟温度测量程序}
\end{figure}

  
\item 创建一个电压输出和采集的程序
  
在 LabVIEW 中,我们可以新建一个空白 VI,并在程序框图窗口中进行操作。
  
对于输入部分,我们需要新建 "DAQmx 创建虚拟通道"(测量I/O $\to$ DAQmx -数据采集 $\to$ DAQmx 创建虚拟通道),并选择模拟输入电压。在其 "物理通道" 接口处右键单击,创建 "输入控件"。接着,我们需要新建 "DAQmx -数据采集"、"DAQmx 读取" 和 "DAQmx 清除任务",并在 "DAQmx 读取" 的 "数据" 接口处右键单击,创建 "显示控件"。最后,我们需要新建 "While 循环"、"等待",在 "等待" 的 "等待时间 ms" 接口处右键单击,创建 "常量",并将其设为 $100$。在 "结束条件" 处右键单击,创建 "停止"。我们还需要调整位置、连线和更改标签。
  
输出部分与输入部分相似,不同之处在于选择 "DAQmx 创建虚拟通道" 并选择模拟输入电压,新建 "DAQmx 写入" 而非 "DAQmx 读取",并在 "DAQmx 写入" 的 "数据" 接口处右键单击,创建 "输入控件"。
  
在前面板中,我们需要修改标签并调整图表位置。对于实验所使用的 ELVIS 仪器面板,我们只需要在打开两个电源后,将 AI 0+ 与 AO 0、AI 0- 与 AIGND 用导线连接即可。然后,我们可以对输出通道/输入通道分别选择 Dev4/ao0 和 Dev4/ai0,改变输出电压,观察测量电压。最后,我们可以保存并关闭文件。
  
前面板和程序框图如图3所示:
  
\begin{figure}[H]
    \centering
    \subfigure[前面板]{\includegraphics[height=5cm]{实验三前面板.jpg}}\hspace{0.5cm}
    \subfigure[程序框图]{\includegraphics[height=5cm]{实验三程序框图.jpg}}
    \caption{实验三: 创建一个电压输出和采集的程序}
\end{figure}
  
\item 利用虚拟仪器测量伏安特性
  
在 LabVIEW 中,我们可以新建一个 VI 文件,并在前面板中新建 "Express XY图",并修改标签,选用"点加线"模式,将横坐标和纵坐标标签分别修改为 "电流(A)" 和 "电压(V)"。接着,我们可以新建 "数值输入控件" $\times 4$,并修改标签并设置单位。我们还可以新建 "数值显示控件" 并修改标签,以及新建 "开关按钮"、"数值显示控件" 和 "数组"。我们可以拖拽使数组成为 $2\times 20$ 以上大小,并将 "数值显示控件" 拖放至数组框内并拖拽成 $2$ 个。
  
编写程序框图的步骤繁琐,需要注意的是,while 循环内部需要添加并行结构,共计 5 个帧,右键边框添加即可。第 1 帧(按编程习惯,为第 0 帧)中的陌生图标是 "进制转换",第 2 帧中的陌生图标是 "索引数组"。while 循环外部上方和右方的陌生图标是 "创建数组"。连接 "待测电阻值" 的陌生图标是 "线性拟合",后者使用输出端 "斜率"。while 循环框上的图表为寄存器,直接右键添加即可。所有 "DAQ助手" 的生成模式均选择 "1 采样(按要求)" 即可,其中有一个并不做要求。
  
接下来需要连接外部电路,如图所示,连接好原型板上导线和电阻即可,其中的蓝色电阻可更换。连接完之后即可运行程序,每次测量只需要运行一次程序即可。在数组箭头处按右键可以将数据导出为 Excel,X-Y图按右键可以将数据导出为位图。在每次实验后更换电阻/稳压二极管,得到三项数据,导出并存储即可。
  
前面板和程序框图如图4所示:
  
\begin{figure}[H]
    \centering
    \subfigure[前面板]{\includegraphics[height=6.5cm]{实验四前面板.jpg}}
    \subfigure[程序框图]{\includegraphics[height=6.5cm]{实验四程序框图.jpg}}
    \caption{实验四: 利用虚拟仪器测量伏安特性}
\end{figure}

\end{enumerate}

\section{实验数据}

\begin{enumerate}

\item 利用虚拟仪器测量伏安特性

在 LabVIEW 中,可以输入相应的参数,然后点击按钮启动程序运行一次。程序会给出测量的电流/电压成对数据,并计算出线性拟合的电阻值。测量得到的电流-电压 ($I$-$U$) 结果如表1,不同待测电路元件对应的输入参数如表2,伏安图像分别为图5中的四个子图:

\begin{table}[H]
    \centering
    \caption{不同待测电路元件对应的输入参数}
    \begin{tabular}{|r|r|r|r|r|r|r|r|}
        \hline
        \multicolumn{2}{|c|}{$1 \unit{k\Omega}$ 电阻} & \multicolumn{2}{c|}{$51 \unit{\Omega}$ 电阻} & \multicolumn{2}{c|}{二极管正向} &\multicolumn{2}{c|}{二极管反向} \bigstrut\\
        \hline
        电压(V) & 电流(A) & 电压(V) & 电流(A) & 电压(V) & 电流(A) & 电压(V) & 电流(A) \bigstrut\\
        \hline
        -0.0012889 & 2.20E-06 & -0.0009667 & -1.03E-06 & -0.0016111 & 5.41836E-06 & -0.0016111 & 2.1961E-06 \bigstrut\\
        \hline
        0.452404 & 0.00046298 & 0.0167557 & 0.0003212 & 0.399236 & -4.24838E-06 & 0.398592 & -1.026E-06 \bigstrut\\
        \hline
        0.904808 & 0.00092698 & 0.0325447 & 0.00065953 & 0.798795 & 5.41836E-06 & 0.798473 & 5.4184E-06 \bigstrut\\
        \hline
        1.3585 & 0.00139098 & 0.0499448 & 0.00098176 & 1.19868 & -1.02613E-06 & 1.19803 & 5.4184E-06 \bigstrut\\
        \hline
        1.81123 & 0.00185499 & 0.0660561 & 0.00131687 & 1.59727 & 1.50851E-05 & 1.59791 & 5.4184E-06 \bigstrut\\
        \hline
        2.26396 & 0.00232866 & 0.0828118 & 0.00164232 & 1.8267 & 0.00170676 & 1.9978 & 5.4184E-06 \bigstrut\\
        \hline
        2.71669 & 0.00279266 & 0.0995675 & 0.00197421 & 1.8963 & 0.00498701 & 2.398 & 8.6406E-06 \bigstrut\\
        \hline
        3.17006 & 0.00325344 & 0.116968 & 0.00229966 & 1.94463 & 0.00847671 & 2.79853 & 2.1961E-06 \bigstrut\\
        \hline
        3.62376 & 0.00371423 & 0.133401 & 0.00263155 & 1.98427 & 0.0120502 & 3.19841 & 2.1961E-06 \bigstrut\\
        \hline
        4.07649 & 0.00417823 & 0.150479 & 0.00295377 & 1.99684 & 0.0133971 & 3.59765 & 5.4184E-06 \bigstrut\\
        \hline
        4.53116 & 0.00463257 & 0.167235 & 0.00329211 & 1.99555 & 0.0132166 & 3.99851 & 2.1961E-06 \bigstrut\\
        \hline
        4.98454 & 0.00509657 & 0.18399 & 0.00361756 & 1.9949 & 0.0131006 & 4.39839 & 5.4184E-06 \bigstrut\\
        \hline
        5.43825 & 0.00555735 & 0.201068 & 0.003943 & 1.99265 & 0.0130362 & 4.79861 & -1.026E-06 \bigstrut\\
        \hline
        5.89196 & 0.00601169 & 0.216857 & 0.00428134 & 1.99233 & 0.0129814 & 5.19882 & 2.1961E-06 \bigstrut\\
        \hline
        6.34631 & 0.0064628 & 0.23458 & 0.00460034 & 1.99168 & 0.0129395 & 5.59774 & 5.4184E-06 \bigstrut\\
        \hline
        6.80035 & 0.00692036 & 0.250047 & 0.00493546 & 1.99072 & 0.0129138 & 5.99796 & 5.4184E-06 \bigstrut\\
        \hline
        7.2531 & 0.00739081 & 0.268413 & 0.0052609 & 1.99039 & 0.0128848 & 6.39817 & 2.1961E-06 \bigstrut\\
        \hline
        7.70521 & 0.00785804 & 0.284525 & 0.0055928 & 1.98975 & 0.0128622 & 6.79807 & -1.026E-06 \bigstrut\\
        \hline
        8.15861 & 0.00832526 & 0.300636 & 0.00592469 & 1.99039 & 0.0128332 & 7.19829 & 5.4184E-06 \bigstrut\\
        \hline
        8.61201 & 0.00878605 & 0.318036 & 0.00625336 & 1.99007 & 0.0128203 & 7.59818 & 2.1961E-06 \bigstrut\\
        \hline
        9.06542 & 0.00924683 & 0.335114 & 0.00658203 & 1.9891 & 0.0128106 & 7.99809 & 5.4184E-06 \bigstrut\\
        \hline
      \end{tabular}
\end{table}


\begin{table}[H]
    \centering
    \caption{不同待测电路元件对应的输入参数}
    \begin{tabular}{|c|c|c|c|}
        \hline
          & $1 \unit{k\Omega}$ 电阻  & $51 \unit{\Omega}$ 电阻  & 二极管 \bigstrut\\
        \hline
        输出电压步长 & 0.5    & 0.05   & 0.4 \bigstrut\\
        \hline
        测量数据点数 & \multicolumn{3}{c|}{20} \bigstrut\\
        \hline
        标准电阻   & \multicolumn{3}{c|}{100} \bigstrut\\
        \hline
        时间间隔   & \multicolumn{3}{c|}{0.02} \bigstrut\\
        \hline
    \end{tabular}
\end{table}


\begin{figure}[H]
    \centering
    \subfigure[$1 \unit{k\Omega}$ 电阻]{\includegraphics[height=5.5cm]{1000欧姆.png}}
    \subfigure[$51 \unit{\Omega}$ 电阻]{\includegraphics[height=5.5cm]{51欧姆.png}}
    \subfigure[二极管正向]{\includegraphics[height=5.5cm]{二极管正向.png}}
    \subfigure[二极管反向]{\includegraphics[height=5.5cm]{二极管反向.png}}
    \caption{不同电路元件测量得到的伏安曲线图}
\end{figure}

从上图我们可以看到,对于$1\unit{k\Omega}$电阻和$51\unit{\Omega}$电阻,伏安特性曲线的$R^2 = 1$,说明线性性非常好。而二极管正向和反向的伏安特性曲线也符合二极管的特性,即正向电压较小时电流较小,随着电压的增大电流迅速增大;而反向很难被击穿,所以电压基本上为0。


根据最小二乘法原理, 假设电压与电流满足线性关系 $U = RI + U_0$ ($U_0$ 为微小干扰值, 根据实验数据明显能看出 $U_0 \approx 0$) 我们可以根据以下公式计算出两个电阻的测量阻值: 
\[
    \hat{R} = \cfrac{\sum\limits_{i=1}^{n} (I_i - \overline{I}) (U_i - \overline{U})}{\sum\limits_{i=1}^{n} (I_i - \overline{I})^2}
\]
对 $1\unit{k\Omega}$ 电阻和$51\unit{\Omega}$ 电阻计算出的阻值分别为 $981.61\unit{\Omega}$ 和 $50.94\unit{\Omega}$,误差分别是$\frac{1000-981.61}{1000}*100\%=1.839\%$与$\frac{51-50.94}{51}*100\%=0.12\%$,与真实的电阻值十分接近,证明实验较为成功。

而两组数据的相关系数计算公式为:
\[
    \rho_{xy} = \cfrac{\operatorname{Cov (X,Y)}} {\sigma_{x}\sigma_{y}} =  \cfrac{\sum\limits_{i=1}^{n} (I_i - \overline{I}) (U_i - \overline{U})}{\sqrt{\sum\limits_{i=1}^{n} (I_i - \overline{I})^2} \sqrt{\sum\limits_{i=1}^{n} (U_i - \overline{U})^2}}
\]

对 $1\unit{k\Omega}$ 电阻和$51\unit{\Omega}$ 电阻计算出的相关系数分别为 $0.99852$ 和 $0.99999$, 在考虑一定的不确定度后, 两个相关系数均大于$0.99$。可见两组数据的线性性非常强, 测量足够准确。

\end{enumerate}

\section{思考题}

\begin{enumerate}    
    \item 虚拟仪器系统与传统仪器有什么区别?请简要说明。
    
    (1) 虚拟仪器使实验者的操作更加自由,例如可以自定义一些功能;而传统仪器的功能相对单一,可设计性远小于虚拟仪器。

    (2) 虚拟仪器可以显示出多种形式的数据形式,例如可以处理转化后的数字信号,也可以展示出图形化的结果;而传统仪器往往没有转化信号的功能,仅能测定某种特定的力学信号,光信号等。
    
    (3) 虚拟仪器与计算机链接,能够便捷传输实验数据,处理起来方便快捷;而传统仪器很难传输数据,甚至需要人工记录。

    (4) 虚拟仪器精确度高、稳定性好,而且占地面积小;二传统仪器精确度稳定性与环境和设备年龄有关,易老化、降低精确度,而且占地面积较大。
    
    \item 本实验内容3中的电压输出和采集哪个先执行?
    
    开始运行后两个程序是并行的,两者的先后顺序仅由编译器来决定。因此,两者开始执行的时间相差并不明显。可以认为两个部分是同时开始执行的。但是考虑实际情况,我们采用先输出后采集的方式,以避免开启电路时的缓冲对测量结果的干扰,使得测量更精准。

\end{enumerate}

\section{实验总结}

在本次实验中,我第一次使用到了LabVIEW 2014平台,并用它完成了三个实验。从开始的生疏到最后的熟练掌握,我对虚拟仪器的使用和便捷性有了更深刻的认识。我认为虚拟仪器能大大简化实验的操作,有利于在减少传统仪器购置成本的同时让同学们和更先进的设备接触。

这个实验也让我见识到了计算机处理大量数据的能力,让我感受到了信息技术对物理实验的巨大帮助。总而言之,这次实验让我受益匪浅,我会继续努力学习,提高自己的实验能力。

\end{document}