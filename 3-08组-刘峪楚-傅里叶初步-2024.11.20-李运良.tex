% 本模板根据中国科学院大学本科生公共必修课程《基础物理实验》Word模板格式编写
% 本模板由Shing-Ho Lin和Jun-Xiong Ji于2022年9月共同完成, 旨在方便LaTeX原教旨主义者和被Word迫害者写实验报告, 避免Word文档因插入过多图与公式造成卡顿. 
% 如有任何问题, 请联系: linchenghao21@mails.ucas.ac.cn
% This is the LaTeX template for experiment report of Experimental Physics courses, based on its provided Word template. 
% This template is completed by the joint collabration of Shing-Ho Lin and Junxiong Ji in September 2022. 
% Adding numerous pictures and equations leads to unsatisfying experience in Word. Therefore LaTeX is better. 
% Feel free to contact us via: linchenghao21@mails.ucas.ac.cn

\documentclass[11pt]{article}

\usepackage[a4paper]{geometry}
\geometry{left=2.0cm,right=2.0cm,top=2.5cm,bottom=2.5cm}

\usepackage{ctex} % 支持中文的LaTeX宏包
\usepackage{amsmath,amsfonts,graphicx,subfigure,amssymb,bm,amsthm,mathrsfs,mathtools,breqn} % 数学公式和符号的宏包集合
\usepackage{algorithm,algorithmicx} % 算法和伪代码的宏包
\usepackage[noend]{algpseudocode} % 算法和伪代码的宏包
\usepackage{fancyhdr} % 自定义页眉页脚的宏包
\usepackage[framemethod=TikZ]{mdframed} % 创建带边框的框架的宏包
\usepackage{fontspec} % 字体设置的宏包
\usepackage{adjustbox} % 调整盒子大小的宏包
\usepackage{fontsize} % 设置字体大小的宏包
\usepackage{tikz,xcolor} % 绘制图形和使用颜色的宏包
\usepackage{multicol} % 多栏排版的宏包
\usepackage{multirow} % 表格中合并单元格的宏包
\usepackage{makecell} % 单元格中换行的宏包
\usepackage{diagbox} % 表格斜线的宏包
\usepackage{pdfpages} % 插入PDF文件的宏包
\RequirePackage{listings} % 在文档中插入源代码的宏包
\RequirePackage{xcolor} % 定义和使用颜色的宏包
\usepackage{wrapfig} % 文字绕排图片的宏包
\usepackage{bigstrut,multirow,rotating} % 支持在表格中使用特殊命令的宏包
\usepackage{booktabs} % 创建美观的表格的宏包
\usepackage{circuitikz} % 绘制电路图的宏包

\definecolor{dkgreen}{rgb}{0,0.6,0}
\definecolor{gray}{rgb}{0.5,0.5,0.5}
\definecolor{mauve}{rgb}{0.58,0,0.82}
\lstset{
  frame=tb,
  aboveskip=3mm,
  belowskip=3mm,
  showstringspaces=false,
  columns=flexible,
  framerule=1pt,
  rulecolor=\color{gray!35},
  backgroundcolor=\color{gray!5},
  basicstyle={\small\ttfamily},
  numbers=none,
  numberstyle=\tiny\color{gray},
  keywordstyle=\color{blue},
  commentstyle=\color{dkgreen},
  stringstyle=\color{mauve},
  breaklines=true,
  breakatwhitespace=true,
  tabsize=3,
}

% 轻松引用, 可以用\cref{}指令直接引用, 自动加前缀. 
% 例: 图片label为fig:1
% \cref{fig:1} => Figure.1
% \ref{fig:1}  => 1
\usepackage[capitalize]{cleveref}
% \crefname{section}{Sec.}{Secs.}
\Crefname{section}{Section}{Sections}
\Crefname{table}{Table}{Tables}
\crefname{table}{Table.}{Tabs.}

\setmainfont{Times New Roman}
\setCJKmainfont{黑体}
\setCJKsansfont{宋体}
\setCJKmonofont{仿宋}
\punctstyle{kaiming}
% 偏好的几个字体, 可以根据需要自行加入字体ttf文件并调用

\renewcommand{\emph}[1]{\begin{kaishu}#1\end{kaishu}}

\newcommand*{\unit}[1]{\mathop{}\!\mathrm{#1}}
\newcommand*{\dif}{\mathop{}\!\mathrm{d}}%微分算子 d
\newcommand*{\pdif}{\mathop{}\!\partial}%偏微分算子
\newcommand*{\cdif}{\mathop{}\!\nabla}%协变导数、nabla 算子
\newcommand*{\laplace}{\mathop{}\!\Delta}%laplace 算子
\newcommand*{\deriv}[2]{\frac{\mathrm{d} #1}{\mathrm{d} {#2}}}
\newcommand*{\derivh}[3]{\frac{\mathrm{d}^{#1} #2}{\mathrm{d} {#3^{#1}}}}
\newcommand*{\pderiv}[2]{\frac{\partial #1}{\partial {#2}}}
\newcommand*{\pderivh}[3]{\frac{\partial^{#1} #2}{\partial {#3^{#1}}}}
\newcommand*{\mcelsius}{\unit{\prescript{\circ}{}C}}
%改这里可以修改实验报告表头的信息
\newcommand{\experiName}{傅里叶光学基础}
\newcommand{\supervisor}{李运良}
\newcommand{\name}{刘峪楚}
\newcommand{\studentNum}{2023K8009929030}
\newcommand{\class}{3}
\newcommand{\group}{08}
\newcommand{\seat}{2}
\newcommand{\dateYear}{2024}
\newcommand{\dateMonth}{11}
\newcommand{\dateDay}{20}
\newcommand{\room}{705}
\newcommand{\others}{$\square$}
%% 如果是调课、补课, 改为: $\square$\hspace{-1em}$\surd$
%% 否则, 请用: $\square$
%%%%%%%%%%%%%%%%%%%%%%%%%%%

\newcommand{\chapter}[2]{\begin{center}\bf\Large{第#1部分\quad #2}\end{center}}

\begin{document}

%若需在页眉部分加入内容, 可以在这里输入
% \pagestyle{fancy}
% \lhead{\kaishu 测试}
% \chead{}
% \rhead{}

\begin{center}
    \LARGE \bf 《\, 基\, 础\, 物\, 理\, 实\, 验\, 》\, 实\, 验\, 报\, 告
\end{center}

\begin{center}
    \noindent \emph{实验名称}\underline{\makebox[25em][c]{\experiName}}
    \emph{指导教师}\underline{\makebox[8em][c]{\supervisor}}\\
    \emph{姓名}\underline{\makebox[6em][c]{\name}} 
    % 如果名字比较长, 可以修改box的长度"6em"
    \emph{学号}\underline{\makebox[10em][c]{\studentNum}}
    \emph{分班分组及座号} \underline{\makebox[5em][c]{\class \ -\ \group \ -\ \seat }\emph{号}} (\emph{例}:\, 1\,-\,04\,-\,5\emph{号})\\
    \emph{实验日期} \underline{\makebox[3em][c]{\dateYear}}\emph{年}
    \underline{\makebox[2em][c]{\dateMonth}}\emph{月}
    \underline{\makebox[2em][c]{\dateDay}}\emph{日}
    \emph{实验地点}\underline{{\makebox[4em][c]\room}}
    \emph{调课/补课} \underline{\makebox[3em][c]{\others\ 是}}
    \emph{成绩评定} \underline{\hspace{5em}}
    {\noindent}
    \rule[8pt]{17cm}{0.2em}
\end{center}

\chapter{一}{阿贝成像与基本空间滤波}

\section{实验目的}

\begin{enumerate}
    \item 掌握一维导轨上光路的调节。
    \item 通过搭建阿贝成像光路和观察不同空间滤波器的效果,体会和理解成像过程、频谱面、谱空间与实空间对应关系、空间滤波、衍射等物理概念。
\end{enumerate}

\section{实验仪器}

\begin{table}[H]
    \centering
    \begin{tabular}{|c|c|}\hline
        组件名称 & 包含器件\\ \hline
        激光器组件& 激光器、棱镜夹持器、一维平移台、宽滑块、支杆和套筒\\ \hline
        扩束镜组件& 凹透镜(Φ$ 6$, $f$-$10$mm )、透镜架、滑块、支杆和套筒\\ \hline
        准直镜组件& 凸透镜(Φ$40$, $f$-$80$mm )、透镜架、滑块、支杆和套筒\\ \hline
        光栅字组件& 光栅字(Φ$40$, $10$线/mm )、滑块、支杆和套筒\\ \hline
        变换透镜组件& 凸透镜(Φ$76$, $f$-$175$mm )、镜架、滑块、支杆和套筒\\ \hline
        滤波器组件& 滤波器(低通、方向滤波)、干板架、滑块、支杆和套筒\\ \hline
        白屏组件& 白屏、干板架、滑块、支杆和套筒\\ \hline
    \end{tabular}
    \caption{实验仪器表}
\end{table}

\section{实验原理}

阿贝成像是傅里叶光学的最基本和最直观的一个展示实验;其很好地体现了波前传递时,透镜操作下频谱面的存在与频谱面的特殊意义。

几何光学把"物"理解为光源,"像"是"物"点发出的光在透镜操作后重新汇聚到的一点(或虚像,是光路反向延长线汇聚的一点),两者一一映射,透镜即实现这一映射的光学器件。而阿贝成像原理指出,除了"物"和"像",成像的过程中还存在一个频谱面(透镜的后焦面),在这里不同物发出的同频率、同偏振方向的光汇聚在如图的三个点上,且满足:
\[
    S_{\pm 1} = \pm F\tan{\theta_i}\text{ , 其中}\sin{\theta_i} = f_i\lambda
\]

$f_i$为余弦光栅的空间频率,$\lambda$为光的波长。我们以这三个点为次级波源,计算他们发射球面波的复振幅,形成干涉,在光屏上的结果即为经典理论计算的结果。

而阿贝成像原理是以光场的概念为核心,将“物”“像”与透镜等看成是操作,这是很类似于物理上将求导看成微分算子$\displaystyle \frac{\mathrm{d}}{\mathrm{d}x}$再“作用”在函数上,从而将微分算子也看成一种操作的,这也是现代光学相较于传统光学的不同。

阿贝成像原理将传统光学的一步分成两步:首先,入射光场被物平面衍射形成一个携带物信息的衍射场,并在透镜的变换操作下,在频谱面形成频谱斑,事实上,这些衍射斑就是“物”信息与光场进行卷积$*$后的结果;然后,这些携带了“物”信息的频谱斑成为新的相干光源发射球面波,并通过光场的进一步传播在像平面实现卷积的逆运算(类似于傅里叶变换与傅里叶逆变换之间的关系),从而完成干涉成像。

本次实验便是通过放置在频谱面的滤波器滤掉多余的光波,仅保留需要的波长范围,以达到改变成像属性的目的。

\begin{figure}[H]
    \centering
    \includegraphics[width=0.8\textwidth]{阿贝成像原理.jpg}
    \caption{阿贝成像原理}
\end{figure}

\section{实验内容}

\subsection{光路布置与调节}

先调整激光发生器、扩束镜的位置,然后再调整其他光学元件的位置。讲义中给出的参考距离大体正确,但仍然要手动微调以达到最佳效果;但注意光学仪器的位置并不等同于其下坐标的位置,需要注意部分仪器的位置有偏差(尤其是后续实验中的白光LED灯,较底座向前突出很多),应当用尺子进一步测量。注意部件应当从激光器开始依次安装,并在确认位置无误后拧紧螺丝。如果在安装过程中发现光路有问题,应当从有问题的器件开始重新安装。装置图如下:

\begin{figure}[H]
    \centering
    \includegraphics[width=0.45\textwidth]{阿贝成像装置图.jpg}
    \caption{阿贝成像装置图}
\end{figure}

安装滤波器的过程中,注意滤波器的位置应在滤波器上激光花样最清晰的位置,此处为频谱面。也可以用光屏来辅助寻找位置。

\subsection{观察“光”栅字的像和频谱}

观察没有滤波状态下的物像,当放大倍数足够大时,可以观察到“光”字的像中间既有横向条纹,也有竖向条纹。这是如果在频谱面上放一个光屏,可以帮助我们分析光谱。如下图所示:

\begin{figure}[H]
    \centering
    \includegraphics[width=10cm]{无滤波.jpg}
    \caption{无滤波时光栅字}
\end{figure}

\subsection{观察方向滤波和低通滤波的实际效果}

观察方向滤波:选择滤波器中的“缝”,在频谱面水平放置,使包括0级光斑在内的一排光斑通过,我们可以观察到“光”的像中间充满竖向条纹。旋转90°,使得“缝”竖直放置,则可以观察到“光”的像中间充满横向条纹。由于两者效果近似,且条纹细小难以观察,实验中需将光屏倾斜放置以放大条纹,便于观察与相机拍摄记录。由于竖向放置滤波器时难以倾斜固定光屏(用手抓光屏容易晃动,照片拍摄时不清晰),故在此仅展示横向滤波的结果。结果如图5。

观察低通滤波:将滤波器中的“孔”放置在频谱面,只让0级光斑通过,我们即可以观察到 “光”的像中间没有条纹,只剩下“光”字轮廓和实心填充。照片如下图所示:

\begin{figure}[H]
    \centering
    \subfigure[滤波器水平缝]{\includegraphics[width=6cm]{水平滤波.jpg}}
    \subfigure[竖向条纹光栅字]{\includegraphics[width=6cm]{竖向条纹.jpg}}
    \subfigure[滤波器竖直缝]{\includegraphics[width=6cm]{竖直滤波.jpg}}
    \subfigure[横向条纹光栅字]{\includegraphics[width=6cm]{横向条纹.jpg}}
    \subfigure[滤波器斜向缝]{\includegraphics[width=6cm]{斜向滤波.jpg}}
    \subfigure[斜向条纹光栅字]{\includegraphics[width=6cm]{斜向条纹.jpg}}
    \subfigure[滤波器孔]{\includegraphics[width=6cm]{低通滤波.jpg}}
    \subfigure[低通滤波光栅字]{\includegraphics[width=6cm]{无条纹.jpg}}
    \caption{方向滤波与低通滤波}
\end{figure}

理论上的“光”字图像和实际观察到的有一定的差异。方向滤波中“光”字的条纹不甚明显,低通滤波的字仍有条纹存在,没有滤波时的图样略有模糊。可能是使用的透镜没有完全校准,导致聚焦并不准确,频谱面没有准确形成或倾斜。

\setcounter{section}{0}

\chapter{二}{光学4F系统成像}

\section{实验目的}

体会和掌握光学4F成像系统的组织和搭建;并在前面阿贝成像实验的基础上,进一步体会更为复杂的光学信息处理。

\section{实验仪器}

\begin{table}[H]
    \centering
    \begin{tabular}{|c|c|}\hline
        组件名称 & 包含器件\\ \hline
        光源组件& 半导体激光器($650$nm)、一维平移台、宽滑块、支杆和套筒\\ \hline
        准直镜组件& 凹透镜($\Phi 6$,$f-9.8$mm)、凸透镜($\Phi 25$,$f-80$mm)、透镜架、滑块、支
        杆和套筒\\ \hline
        调制物组件& 物板、干板架、滑块、支杆和套筒\\ \hline
        变换透镜组件& 凸透镜($\Phi 40$, $f-175$mm )、镜架、滑块、支杆和套筒\\ \hline
        滤波器组件& 滤波器(低通、方向滤波)、精密平移台、干板夹、滑块、支杆和套筒\\ \hline
        白屏组件& 白屏、干板架、滑块、支杆和套筒\\ \hline
    \end{tabular}
    \caption{实验仪器表}
\end{table}

\section{实验原理}

4F 图像处理系统使用两个透镜依次实现傅里叶变换和反傅里叶变换的光学操作,把成像要素与频谱操作要素分离开,是一种可控性、保真性、稳定性更好的相干光学处理系统。通过使用两个透镜,依次实现傅里叶变换和反傅里叶变换的光学操作,把成像要素与频谱操作要素分离开,频谱面位于两个透镜的中间,对成像的干扰小。连续进行两次傅里叶变换操作会为原函数添加一个负号。一束平行光照射前焦面处的透明物体,产生待处理的图像,在第一个透镜的后焦面上得到物函数的频谱;而频谱面也是第二个透镜的前焦面,于是在第二个透镜的后焦面上得到第二次傅里叶变换,得到了原函数的倒像。可以在频谱面上插入空间滤波器,改变频谱函数,处理输入信号。

\section{实验内容}

逐个安装并调整光学器件的位置。讲义中给出的参考距离基本正确,可以按照该距离安装;注意同上一部分。最终得到的像是一个和物等大反向的像。但因自制图像后夹住滤波器的底座不足,故未添加滤波器。因为没有拿剪刀,所以使用了抽屉里现成的纸片。结果如下:

\begin{figure}[H]
    \centering
    \subfigure[自制4F成像装置图]{\includegraphics[height=6cm]{光学4F成像装置.jpg}}
    \subfigure[实验室纸片的4F成像]{\includegraphics[height=6cm]{光学4F成像.jpg}}
    \caption{光学4F系统成像}
\end{figure}

可以看到,光屏上确实出现了清晰等大的倒像,实验成功。

\setcounter{section}{0}

\chapter{三}{假彩色编码}

\section{实验目的}

在基本空间滤波的基础上,进一步体会光栅衍射的色散效果和选频滤波操作,掌握$\theta$调制假彩色编码的选频滤波和色散选区滤波的原理;并利用提前预制分区信息的光栅图案,实现该图像的假彩色编码。

\section{实验仪器}

\begin{table}[H]
    \centering
    \begin{tabular}{|c|c|}\hline
        组件名称 & 包含器件\\ \hline
        光源组件& 白光LED、一维平移台、宽滑块、支杆和套筒\\ \hline
        准直镜组件& 凸透镜($\Phi 40$,$f-80$mm)、透镜架、滑块、支杆和套筒 \\ \hline
        调制物组件& 天安门光栅($100$线/mm)、干板架、滑块、支杆和套筒\\ \hline
        变换透镜组件& 凸透镜($\Phi 76$,$f-175$mm)、镜架、滑块、支杆和套筒\\ \hline
        滤波器组件& 滤波器、干板架、滑块、支杆和套筒\\ \hline
        白屏组件& 白屏、干板架、滑块、支杆和套筒\\ \hline
    \end{tabular}
    \caption{实验仪器表}
\end{table}

\section{实验原理}

用白光光源照明一个事先预制不同取向光栅的天安门图案,然后分别使用颜色滤波器和自制的空间选色滤波器,来实现天安门图像的选区假彩色编码。

在天安门光栅中,天空、天安门和草地三个区域分别预设了不同方向的光栅刻线,且每个区域对应不同的颜色:蓝色、红色和绿色。白光光源照射到透明的天安门上时,会发生衍射,产生的不同颜色光会向不同方向传播,并通过透镜聚焦到频谱面上,形成彩色的衍射图案。由于光栅上不同区域的刻线方向各异,衍射图案会沿着三个不同的方向展开,呈现带状彩色花纹。通过选择三个不同方向的彩色滤片(红、绿、蓝)或在白纸上打出不同位置的孔洞,可以选取并过滤不同的颜色,实现区域的彩色编码。最终,在光屏上我们将看到蓝色的天空、红色的天安门和绿色的草地。

\section{实验内容}

\subsection{光路布置和调节}

按照讲义参数布置光路,并结合实际情况适量微调。

实验室提供的滤光片效果如下:

\begin{figure}[H]
    \centering
    \subfigure{\includegraphics[height=5cm]{正确安装滤片.jpg}}
    \caption{假彩色编码调制的天安门图样}
\end{figure}

可以看出蓝天、绿草、红色的天安门十分清晰,实验比较成功。

另外,实验中还可以使用自制的空间选色滤波器,将滤波器放在频谱面上,可以实现选区假彩色编码。实验结果如下:

\begin{figure}[H]
    \centering
    \subfigure[自制的滤波器]{\includegraphics[height=5cm]{自制滤波器.jpg}}
    \subfigure[假彩色编码自己调制的天安门图样]{\includegraphics[height=5cm]{自制滤波片图样.jpg}}
    \caption{假彩色编码自己调制的天安门图样}
\end{figure}

\setcounter{section}{0}

\chapter{四}{衍射实验}

\section{实验内容}

\subsection{夫琅和费衍射演示实验}

将分划板放入光路中,看衍射图样的变化,根据所学判断衍射图样与分划板参数是否一致。这一部分实验是观察性质的,因此将实验结果直接展示。实验结果如下:

这是分划板1的实验结果:

\begin{figure}[H]
    \centering
    \subfigure[小孔图样]{\includegraphics[height=5cm]{小孔.jpg}}
    \subfigure[小屏图样]{\includegraphics[height=5cm]{小屏.jpg}}
    \subfigure[单丝图样]{\includegraphics[height=5cm]{单丝.jpg}}
    \subfigure[单缝图样]{\includegraphics[height=5cm]{1单缝.jpg}}
    \caption{夫琅和费衍射演示(分划板1)}
\end{figure}

这是分划板2的实验结果:

\begin{figure}[H]
    \centering
    \subfigure[多缝图样]{\includegraphics[height=5cm]{DF2:9缝a= 0.06 d= 0.1×9.jpg}}
    \subfigure[双缝图样]{\includegraphics[height=5cm]{SF3:a= 0.06 d= 0.10.jpg}}
    \subfigure[单缝图样]{\includegraphics[height=5cm]{单缝.jpg}}
    \subfigure[矩孔衍射图样]{\includegraphics[height=5cm]{矩孔衍射.jpg}}
    \subfigure[双孔图样]{\includegraphics[height=5cm]{双孔.jpg}}
    \subfigure[纵向光栅图样]{\includegraphics[height=5cm]{GS2:纵向50条mm;.jpg}}
    \subfigure[纵横光栅图样]{\includegraphics[height=5cm]{GS1:纵横均为50条mm.jpg}}
    \caption{夫琅和费衍射演示(分划板2)}
\end{figure}

\subsection{光栅衍射测量光栅常数实验}

将透射光栅放入光路中,看衍射光斑图样,根据光栅方程算出光栅常数$d$, 判断与已知光栅刻缝数是否一致。

光栅方程为: $d\sin{\theta} = m\lambda \qquad m=0,\pm 1,\pm 2,\,\cdots$

光栅常数$d$为相邻两缝的中心距离,即光栅每毫米刻缝数的倒数,$\theta$表示从干涉图样中心到第$m$级极大之间的夹角,$\lambda$表示光的波长,$m$表示级次。

测量得到$\sin \theta = 0.0267$. 又已知$\lambda=650$nm, 取光栅方程中的$m=4$,代入公式得到:

\[
    d=\frac{m\lambda}{\sin{\theta}}=9.74\times 10^{-5}\text{ m}
\]

\subsection{光栅光谱仪测光谱实验}

使用手持式光栅光谱仪和 SpectraSmart 软件测量白光光谱,记录中心波长、峰值强度与线宽,判断与经验值是否一致。

通过计算机测量并处理后,得到的白光光谱如下图所示:

\begin{figure}[H]
    \centering
    \includegraphics[width=0.8\textwidth]{白光光谱.jpg}
    \caption{白光光谱}
\end{figure}

从图中我们可看出:白色LED的灯光在蓝光和黄绿光区域集中,蓝绿光则极少。

\section{思考题}

\begin{enumerate}
    \item (a) 阿贝成像中,当“缝”与光栅方向夹角 45 度放置滤波时,会有何效果?(b) 前面实验中,我们使用低通滤波(仅让 0 级斑通过)实现了光栅格子信息的消除;如何做个高通滤波的例子?应该如何实现和它的效果是什么?
    
    (a) 在成的像中会产生斜向45度的条纹。

    (b) 实现:滤波器应当滤去 0 级点而使其他点通过。比如在滤波器中央为实心,四周进行镂空处理。效果:“光”字内依旧充满横竖相间的条纹,但此时红色条纹与暗条纹的位置与之前恰好相反。

    \item 观察 4F 系统成像与阿贝成像时单透镜成像的区别是什么?
    
    阿贝成像的基础是光学衍射理论,受限于透镜的数值孔径(NA)和光波长,导致分辨率受衍射极限限制。只记录物体在空间域的复合信息,无法直接区分不同的频率成分。而4F系统是一种基于傅里叶光学的成像方法,在频谱平面上对高频信息(高分辨率细节)具有更好的捕捉能力,可通过调整光阑或滤波器保留或增强某些频段,实现更高的清晰度和信息保留率。

    \item 假彩色编码实验中使用的天安门城楼光栅本身中的城楼的窗户和门洞都是透光的,但是为什么经过所提供的假着色滤波处理后所成的像中这些窗户和门洞是黑色的?有方法验证你的解释吗?
    
    通过窗户和门洞进入的光是直接透射的,这部分信息集中在频谱面的中心区域。然而,由于滤波器的中心没有开孔,这些光无法穿过滤波器继续传播,因此在白屏上对应的位置呈现为黑色。验证方法:在频谱面中间加一个小孔,使得中心光透过,即可在白屏上看到窗户和门洞呈现出白色。

    \item 从夫琅和费衍射演示实验得到哪些规律? 
    
    (1) 衍射图样的强度分布由孔或障碍物的几何形状和大小决定。

    (2) 光栅的缝的数量越多,主极大值越窄越亮,条纹的分辨率越高。

    (3) 夫琅和费衍射是光波动性的直接证据,是光的波动性质的重要实验。

\end{enumerate}

\section{实验总结}

本次实验是以动手和观察为主,计算量和数据量都比较小。通过动手做实验,让光学课本上深奥的知识变得直观,加深了我对这些光学现象的理解。而且光学实验对精确度的要求很高,在摆放光学器具时一定要十分仔细。有时候只要轻轻碰一下,光路就会发生变化,这可能导致了实验结果的不准确,所以在实验中一定要小心谨慎。

总而言之,本次实验是对我理论和实验能力的一次综合考察与锻炼,还学习到了很多新的知识,比如阿贝成像、4F 成像的原理,让我受益匪浅。

\end{document}